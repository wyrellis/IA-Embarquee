\documentclass[11pt]{report} 

\usepackage{projectreport}
\usepackage{fancyhdr}
\usepackage{eurosym}
\pagestyle{fancy}
\renewcommand\headrulewidth{1pt}
\fancyhead[L]{\quad Jalon 1 - Gestion de projet}
\fancyhead[R]{\today}
\fancyfoot[L]{Marie Vialle}
\fancyfoot[R]{Version 5}

\newcommand{\name}{Alexandre Berard, Julien Liottard, Marie Vialle}
\newcommand{\course}{CARI Electronic}

\newcommand{\projecttitle}{IA Embarquée}

\begin{document}

\maketitle
\setcounter{page}{0}
\newpage


\LARGE Introduction
\newline
\newline \large CARI electronic est une entreprise spécialisée dans la fabrication de matériel électronique mélangeant PCB (Printed Circuit Board) et matériaux complémentaires pour former le hardware. Cette entreprise est située à Valence dans le parc du 45ème parallèle. L’entreprise peut assurer plusieurs étapes de production :
    \begin{itemize}
        \item Le câblage,
        \item La mécanique
        \item Les tests grâce à des bancs de test (tests en série, température, vibrations, …) pour tester la fiabilité.  
    \end{itemize}
    De plus, environ 50 \% de la production est orientée dans le secteur aéronautique. Pour finir, l’entreprise base ses productions sur trois facteurs : la fiabilité, la viabilité et la nécessité car ce sont les facteurs de réussite d’un produit.

\newpage


\pagenumbering{arabic}


\paragraph{\LARGE Sommaire}
\begin{enumerate}
	\item Présentation du projet
	\begin{itemize}
	    \item Contexte
	    \item Présentation
	    \item Visualisation du résultat attendu / demandé
	    \item Fonctionnalités demandées
	\end{itemize}
	\item Présentation de l'équipe et comité de pilotage
    	\begin{itemize}
    	    \item Description détaillée des rôles
    	    \item Attribution des rôles
    	    \item Les backups
    	    \item Enseignant tuteur
    	    \item Comité de pilotage
    	\end{itemize}
	\item Dates importantes
    	\begin{itemize}
    	    \item Projet du semestre 3
    	    \item Projet du semestre 4
    	\end{itemize}
    \item Annexes et descriptions
        \begin{itemize}
    	    \item Planning
    	    \item Suivis des coûts
    	\end{itemize}
\end{enumerate}

\newpage

\begin{enumerate}
	\item \LARGE Présentation du projet
	\begin{itemize} \large
	    \item \large Contexte 
	    \newline L'intelligence artificielle est de plus en plus utilisée de nos jours. Le but est de l'utiliser dans un contexte utile à certaines entreprise utilisant des machines électroniques afin de discerner une passe dans le réseau électrique (coupure de courant, tension anormalement élevée, démarrage défectueux, etc.) de manière automatique. Cela sera intéressant pour notre apprentissage et nos connaissances envers de nouveaux outils actuels de l'informatique.
	    
	    \item Présentation du projet
	    \newline Pour répondre au besoin du client, il nous est demandé d'utiliser une carte STM32 avec le logiciel STM32CubeMx (où nous utiliserons le langage C), qui permet d'utiliser des outils AI et de les intégrer. Nous avons cependant le choix en ce qui concerne le framework de réseau de neurones artificiels. Le système d'exploitation sur lequel nous allons travailler est Windows. Pour finir, un de nos objectifs principal est de procurer au client un outil qui permet d'analyser des données en temps réel
	    
	    \item Visualisation du résultat attendu / demandé
	    \newline Le projet final pourrait être imagé par un PC connecté à une carte. Cette même carte serait branchée à des câbles reliant une alimentation et un appareil. La tension et le courant seraient testés sur ces câbles lorsque l'alimentation est branchée et débranchée afin de de réceptionner différentes données. Le programme d'analyse serait donné par le PC utilisant STMCube32.
	    
	    \item Fonctionnalités demandées
	    \newline L'objectif de ce projet est principalement d'effectuer des analyses de tension et d'intensité entre une alimentation et un appareil lors de la mise sous tension ou hors tension de ce dernier, puis de le comparer avec les caractéristiques apprises par l'IA d'un appareil sans défaut. Par la suite, la carte pourrait notifier une anomalie de fonctionnement de l'appareil avec des voyants (l'IHM n'est pas un objectif principal mais cela reste une solution). Ces analyses pourraient notamment être représentées sous forme de courbes avec la tension (U) et l'intensité (I) en fonction du temps. 
	    \newline Un plus serait de pouvoir détecter lorsqu'il y a panne, de quel type de panne il s'agit, afin d'aider le technicien à la corriger.
	\end{itemize}
	
	\item \LARGE Présentation de l'équipe et comité de pilotage
    	\begin{itemize} \large
    	    \item Description détaillée des rôles
    	    \newline Chef de projet : le responsable planning est la personne qui gère la préparation du projet, la coordonne, anime l'équipe et fait les bilans.
    	    \newline Responsable planning : la personne qui gère le planning va mettre en place un Gantt Project et un diagramme de Pert associé. Il doit être en adéquation avec le responsable des coûts afin que les temps de travail soient égaux. C'est un rôle d'organisation.
    	    \newline Responsable des coûts : le responsable des coûts doit mettre à jour le tableau des coûts chaque semaine afin d'être précis dans l'économie liée au projet. Il doit communiquer avec le responsable planning. C'est aussi un rôle d'organisation.
    	    \newline Responsable de communication : C'est le porte parole du groupe avec le client, le tuteur ou le comité de pilotage. 
    	    \newline Responsable de documentation : Au début du projet, il met en place les formats de documents (ici création des fichiers textes en \LaTeX \space et transmission en PDF et excel - Les images seront définie en PNG). Au cours du projet, il vérifie la mise en page des documents afin d'avoir une communication écrite homogène. 
    	    
    	    \item Attribution des rôles
        	    \newline
        	    \newline
    	        \begin{tabular}{|c|c|}
    	            \hline
    	             Marie Vialle & Chef de projet, (communication et documentation) \\
    	             \hline
    	             Alexandre Berard & Responsable Planning \\
    	             \hline
    	             Julien Liottard & Responsable des coûts \\
    	             \hline
    	        \end{tabular}
    	        \newline
\newpage     	        
    \item Les backups
	    \newline 
	    \newline
	    \begin{tabular}{|c|c|}
            \hline
             Nom du responsable & Nom du backup \\
            \hline
             Marie Vialle & Alexandre Berard \\
             \hline
             Alexandre Berard & Julien Liottard \\
             \hline
             Julien Liottard & Marie Vialle \\
             \hline
        \end{tabular}
        \newline
    
    \item Enseignant tuteur
    \newline Notre tuteur est Monsieur Lagrèze. Il travaille à l'IUT de Valence et au laboratoire LCIS de Grenoble-INP.
    
    \item Comité de pilotage
        \newline Monsieur Charensol : aide pour l'intelligence artificielle et l'analyse des courbes;
        \newline Monsieur Lagrèze : aide pour la programmation en C et pour l'électronique embarquée;
        \newline Monsieur Occello : aide pour les spécifications et les éventuels diagrammes.

    	\end{itemize}
	\item \LARGE Dates importantes
    	\begin{itemize}  \large
    	    \item Projet du semestre 3
    	    \begin{itemize} 
    	        \item Semaine du 7 octobre 2019
    	            \newline Pendant la semaine du 7 octobre, nous avons rencontré le client, Monsieur Ceysson, responsable du bureau d'étude de CARI Electronic. Nous avons ensuite commencé à produire une version alpha du cahier des charges et de la note de cadrage. 
    	            
        	    \item Vendredi 18 octobre 2019 
            	    \newline Le vendredi 18 octobre, nous allons rendre ce premier livrable, qui présente l'équipe, le projet et le comité de pilotage. Nous ajouterons en pièce jointe une ébauche du planning (nous n'avons pas encore toutes les dates précises et la procédure de production que nous allons suivre), et le suivi des coûts.
            	    
        	    \item Semaine du 6 novembre 2019
        	        \newline La semaine du 6 novembre, nous rendrons le livrable des besoins fonctionnels à monsieur Occello. Nous allons donc identifier les objectifs stratégiques et utilisateurs, faire les cas d'utilisation.
        	        
        	    \item Vendredi 6 décembre 2020
        	        \newline Le vendredi 6 décembre, nous remettrons notre plan de travail à notre tuteur, monsieur Lagrèze. Ce livrable contiendra le cahier des charges avec le recueil des besoins, l'analyse théorique du problème posé, les méthodes adaptées et des études de solutions techniques.
        	        
        	    \item Jeudi 12 décembre 2019
        	        \newline La première soutenance se déroulera le jeudi 12 décembre. Nous devrons y présenter notre cahier des charges, et plus précisément le recueil des besoins, l'analyse théorique du problème posé, les méthodes adaptées et l'étude des solutions techniques.
        	    
    	    \end{itemize}
    	    \item Projet du semestre 4
    	    \begin{itemize} \large
        	    \item Vendredi 14 février 2020
        	        \newline Le vendredi 14 février, nous rendrons notre état d'avancement de gestion de projet. Nous y joindrons le planning et le suivi coûts (propre à la seconde partie du projet).
        	        
        	    \item Vendredi 20 mars 2020
        	        \newline Le vendredi 20 mars, nous rendrons de nouveau un état d'avancement, semblable à celui du 14 février 2020.
        	        
        	    \item Mardi 24 mars 2020
        	        \newline Le mardi 24 mars, nous remettrons nos rapports à notre client et monsieur Lagrèze.
        	        
        	    \item Vendredi 27 mars 2020
        	        \newline La soutenance de projet se déroulera le vendredi 27 mars. Nous y présenterons notre projet grâce à une démo.
    	    \end{itemize}
    	\end{itemize}
\newpage
        \item Annexes et descriptions\LARGE
            \begin{itemize}\large
        	    \item Planning
        	        \newline Sur le Gantt, nous avons défini un code couleur :
            	    \newline
            	    \begin{tabular}{|c|c|}
                        \hline
                         Rouge & Documentation et recherche \\
                        \hline
                         Violet & Livrables de gestion de projet \\
                         \hline
                         Bleu & livrables de méthodologie de création d'application \\
                         \hline
                    \end{tabular}
                    \newline
                    \newline En ce qui concerne le planning, nous avons défini que nous travaillerons 9h par semaine et par personne, afin de respecter les 100h de travail personnel indiqué dans le PPN informatique.
                    \newline Le projet a commencé le 30 septembre 2019, et nous avons tout d'abord décidé de nous informer sur le sujet qui nous est proposé. Le premier rendez-vous avec le client a eu lieu le 7 octobre 2019 et nous a permis de commencer le travail sur les livrables futurs. Nous travaillons aussi en parallèle sur les différentes tâches que nous devrons accomplir lors du projet (que nous verrons au S4).
                    En clair, nous découpons notre projet en deux parties : la partie livrables de projet, et la partie travail sur le projet en concret (programmation, au S4).
                    D'une part, nous effectuerons les comptes-rendus des réunions, les livrables de gestion de projet et la note de cadrage (3 jalons). Nous avons décidé d'effectuer ces tâches en parallèle sachant que le temps est limité. S'ensuit l'analyse des besoins fonctionnels, un livrable intermédiaire, puis la préparation du plan à livrer au tuteur avant l'oral (3 jalons aussi). Nous nous permettons de réaliser ces tâches à la suite des unes des autres car le temps nous le permet.
                    D'autre part, (côté réalisation du projet), nous travaillons premièrement sur la documentation sur les frameworks proposés dans le sujet du projet, les IDE dont nous avons discuté lors du rendez-vous et les micro-contrôleurs. Nous procéderons à un choix sur les différents IDE etc., il sera donc intéressant d'intégrer un outil de prise de décisions dans le rapport à rendre avant l'oral à ce moment là. Nous prendrons connaissance des outils choisis par la suite, mais nous arrêterons lorsque nous commencerons à travailler sur la préparation de l'oral et la préparation du rapport final, cette tâche nous permettra d'anticiper sur notre projet du S4.
                    \newline Pour la dernière tâche, qui est la finalisation de la préparation de la soutenance, sa durée s'étale sur deux semaines. Nous avons préféré 'lisser' la répartition en horaire de travail qui revient au même au final (en terme de somme d'heure totale) car nous préférons travailler de manière modérée et arrangée sur la fin du projet, même si le rythme augmente brusquement.
                    
        	    \item Suivis des coûts
        	        \newline Depuis le début du projet, nous avons fait une réunion d'une heure le lundi 7 octobre, nous n'avons donc pas pu commencer les spécifications du projet avant. Le coût baisse donc légèrement. 
                    A partir de maintenant, nous allons rédiger les spécifications, la note de cadrage et les jalons, nous allons donc récupérer notre retard.
                    \newline Nous allons travailler 1h de plus par personne du 28 octobre au 3 novembre afin de nous documenter sur les logiciels et hardware proposés pour le projet. 
                    \newline Avec le coût horaire technicien (50\EUR de l'heure), les heures des experts et le coûts des ordinateurs, nous obtenons un coût de revient initial de 17 000€. En prenant en compte les aléas, les frais généraux, nous en déduisons un prix de vente minimal de 24 684\EUR (avec 10\% de marge). Ainsi, avec un prix de vente de 27 000\EUR on a négocié un prix de 26 000\EUR avec le client.
        	\end{itemize}
    \end{enumerate}


\end{document}
